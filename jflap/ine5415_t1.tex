\documentclass{article}

\usepackage{amsthm, amsmath, listings, url}
\usepackage[utf8]{inputenc}
\usepackage{graphicx}
\usepackage[colorlinks=true,
            urlcolor=blue]{hyperref}
\usepackage[a4paper,
            left=20mm,
            right=20mm,
            top=20mm,
            bottom=20mm]{geometry}
 
\renewcommand*\theenumi{\textbf{\alph{enumi}}}

\begin{document}

\begin{center}
    \section*{INE5415 - Teoria da Computação (2015/1)}
    \textbf{Trabalho 1 - Máquinas de Turing \\
    Antonio Luiz Rosa Teixeira, Gustavo Zambonin}
\end{center}

\subsection*{Questão 1}
\textit{L(M)} = \{$0^{2^{n}}$ p/ $n \geq 0$\}
\begin{itemize}
    \item \textbf{Descrição do funcionamento}: A máquina, primeiramente, marca um símbolo vazio no começo da fita para que saiba o seu início. Depois, conta os zeros aos pares, marcando-os e voltando ao início da fita. Se ainda existirem zeros desmarcados, o processo se repetirá, mas apenas marcando um a cada quatro zeros, e assim por diante, respeitando as potências de 2. O processamento da máquina força o estado de rejeição na primeira marcação dos zeros se não encontrar uma entrada de tamanho $2^{n}$.
    \item \textbf{Codificação da máquina}: \\
    \includegraphics[scale=0.5]{questao1_ss.png}
    \item \textbf{Testes realizados}: \\ \\
    \includegraphics[width=\textwidth]{questao1_inputs.png}
\end{itemize}
\newpage
\subsection*{Questão 2}
\textit{L(M)} = \{$a^{i}b^{j}c^{k}$ \vert $ i \times j = k $ e $i, j, k \geq 1$\}
\begin{itemize}
    \item \textbf{Descrição do funcionamento}: A máquina marca um A e a quantidade inteira de letras B, e a mesma quantidade de letras C, fazendo uma operação similar à soma de multiplicações triviais. Ao final da marcação de letras C, as letras B são desmarcadas e a próxima letra A é marcada, e assim por diante. Se a multiplicação não apresentar seu resultado correto, lembrando que todas as letras precisam aparecer pelo menos uma vez, a máquina rejeitará a entrada.
    \item \textbf{Codificação da máquina}: \\
    \includegraphics[scale=0.5]{questao2_ss.png}
	\item \textbf{Testes realizados}: \\ \\
    \includegraphics[width=\textwidth]{questao2_inputs.png}
\end{itemize}
\newpage
\subsection*{Questão 3}
\textit{L(M)} = \{$\#x_{1}\#x_{2}\#...\#x_{n}$ \vert $ x_{i} \in \Sigma = \{0,1\}^{*}$ e $x_{i} \neq x_{j}$ para cada $i \neq j$\}
\begin{itemize}
    \item \textbf{Descrição do funcionamento}: A máquina compara $x_i$ e $x_j$, $\forall i\neq j$, exceto pela palavra vazia, garantida no início do procedimento. Após esta garantia, um $x_i$ é fixado e comparado com $x_{i+1}$, $x_{i+2}$, ..., $x_n$. Depois do final dessa comparação, a próxima subpalavra, $x_{i+1}$, é fixada e comparada com os elementos posteriores, separados pelo \#, até que este não seja mais encontrado na entrada, significando o fim da mesma.
    \item \textbf{Codificação da máquina}: \\
    \includegraphics[width=\textwidth]{questao3_ss.png}
    \item \textbf{Testes realizados}: \\ \\
    \includegraphics[width=\textwidth]{questao3_inputs.png}
\end{itemize}
\end{document}
